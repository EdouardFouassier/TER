\documentclass[11pt]{article}
\usepackage[T1]{fontenc}
\usepackage[utf8]{inputenc}
\usepackage[right=2.5cm, left=2.5cm, bottom=4cm, top=3cm]{geometry}
\usepackage[french]{babel}
\usepackage{textcomp}
\usepackage{graphicx}
\usepackage{mathtools,amssymb,amsthm}
\usepackage{lmodern}
\usepackage{multirow}
\usepackage{array}
\usepackage{algorithm}
\usepackage{algorithmic}

\title{\vspace{13em}{\huge TER}}
\author{Rémi Navarro - 21401257\\ Edouard Fouassier - 21400750}

\begin{document}

\pagenumbering{gobble}
\clearpage
\maketitle\vspace{13em}
\newpage
\tableofcontents
\newpage
\clearpage
\pagenumbering{arabic}

\section{Introdution}
Dans le cadre du module TER du S2 Master Informatique à l’UVSQ, nous avons eu l’occasion de réaliser un projet sous la direction de Mr Yann Strozecki et Mael Guiraud.
Nous avons choisi, parmi les sujets proposés, le sujet "Algorithme glouton de remplissage" car c'est une sujet qui demande une bonne compréhension de l'algorithmique ce qui nous a beaucoup intéressé.

De nos jour les échanges par les différents réseaux sont centralisés dans des datacenters ou cloud.
Pour gagner en efficacité il faut minimisé la latence lors de l'envoie d'un message vers un cloud.

L'objectif de ce projet est de concevoir et comparer des algorithmes gloutons qui permettent de placer au mieux des tâches periodiques avec des contraintes portant sur les paires de tâches.
Pour cela nous utilisons un modèle où les tâches sont envoyés periodiquement et le temps entre l'envoie et la reception est fixe.
Dans ce modèle il y a deux periodes de taille P, l'envoie du tâches est placé sur la première periode et la réception sur la seconde après un delai.
Il faut donc réussir a placer un maximum de tâches dans la periode.

\section{Structures de données}

Dans un premier temps nous utilisions les structures suivantes :\\\\
\indent \textbf{Task} \{ \\
    \indent \indent entier num \indent\indent//Le numero de la tache pour l'identifier\\
    \indent \indent entier delay \indent\indent//Le delai avant le retour sur la dexime periode\\
    \indent \indent entier cycle[2] \indent//Le temps occupé sur la première et deuxième periode\\
    \indent \indent entier place \indent\indent//La place dans la periode, son début sur la première periode,\\
    \indent \indent \indent \indent \indent \indent \indent \indent si la tache n'est pas placé cette valeur est à -1..\\
\indent\}
\\\\
\indent \textbf{Chaine} \{ \\
    \indent \indent Task t   \indent \indent \indent //une tache\\
    \indent \indent chaine $\uparrow$next \indent //la tache suivante.\\
\indent\}
\\\\
\indent La période était stockée dans deux tableaux d'entier, nous écrivions le numéro de la tache dans la ou les case(s) qu'elle occupait.\\
Mais comme seul les espaces disponibles de la periode nous interesse, cette structure n'était pas optimale.\\
Nous sommes donc passé a une structure représentant les espaces libres de la periode sous forme d'une chaine.
\begin{center}
    \begin{tabular}{|l|c|c|}			
    \hline                                              & \textbf{Structure initiale}   & \textbf{Nouvelle structure} \\
    \hline 	Periode initiale de taille 10               & [0,0,0,0,0,0,0,0,0,0]         &(0,9)		     \\
    \hline 	Placement d'une tache de taille de en 5 	& [0,0,0,0,0,1,1,0,0,0] 		& (0,4)$\rightarrow$(6,9)   \\
    \hline
    \end{tabular}\vspace{1em}
\end{center}
\noindent Les taches n'étant plus stockées dans une liste mais dans un tableau, cela a permis de réduire la mémoire utilisée et d'augmenter la taille des tests effectués.\\\\
La structure Periode est utilisée pour représenter les intervalles disponibles d'une periode.\\\\
\indent \textbf{Periode} \{ \\
    \indent \indent entier begin    \indent \indent//Le debut de la periode libre\\
    \indent \indent entier end \indent \indent //La fin de la periode libre.\\
    \indent \indent Periode $\uparrow$next \indent //La periode libre suivante.\\
\indent\}
\\\\
La structure Tasktab représente un tableau de tâches.\\\\
\indent \textbf{Tasktab} \{ \\
    \indent \indent Task tab    \indent \indent//Le debut de la periode libre\\
    \indent \indent entier taille \indent   //La fin de la periode libre.\\
\indent\}

\section{Algorithmes}

\begin{algorithm}
    \caption{FirstFit}
    \begin{algorithmic}
    \REQUIRE Tasktab, PeriodeMax
    \FOR{chaque Task dans Tasktab}
        \FOR {$i \leftarrow  0$ to PeriodeMax}
         \IF {task entre dans la periode aller et t entre dans la periode retour après le delay}
            \STATE $task.place \leftarrow i$
         \ENDIF
        \ENDFOR
    \ENDFOR
    \RETURN Tasktab
    \end{algorithmic}
\end{algorithm}

\begin{algorithm}
    \caption{AlgoLourd}
    \begin{algorithmic}
    \REQUIRE Tasktab, PeriodeMax
    \STATE $min \leftarrow PeriodeMax$
    \STATE $taskMin \leftarrow 0$
    \STATE $libreMin \leftarrow 0$
    \FOR{chaque Task}
        \FOR {chaque Task t dans Tasktab}
        \STATE $compteur \leftarrow 0$
        \STATE $libre \leftarrow 0$
            \FOR {$i \leftarrow  0$ to PeriodeMax}
                \IF {t entre dans la periode aller et t entre dans la periode retour après le delay}
                    \STATE $compteur \leftarrow compteur + 1$
                    \STATE $libre \leftarrow i$
                \ENDIF
            \ENDFOR
            \IF {compteur <= compteurMin}
                \STATE $compteur \leftarrow compteurMin$
                \STATE $taskMin \leftarrow t$
                \STATE $libreMin \leftarrow libre$
            \ENDIF
        \ENDFOR
        \STATE $taskMin.place \leftarrow libreMin$
    \ENDFOR
    \RETURN Tasktab
    \end{algorithmic}
\end{algorithm}


\begin{algorithm}
    \caption{AlgoSuperLourd}
    \begin{algorithmic}
    \REQUIRE Tasktab, PeriodeMax
    \STATE $cptAvant[tasktab.nbTask]$
    \STATE $gene[tasktab.nbTask]$
    \FOR{chaque Task}
        \STATE $cptAvant[] \leftarrow cptplace()$ (cptplace() permet de compter le nombre de places disponibles pour chaque tâche)
        \STATE $cptApres[tasktab.nbTask]$
        \FOR {chaque Task t dans Tasktab}
            \STATE $gene[t] \leftarrow 0$
            \STATE Place t dans la période au premier endroit disponible
            \STATE $cptApres[] \leftarrow cptplace()$ //cptplace() permet de compter le nombre de places disponibles pour chaque tâche
            \STATE $gene[t] \leftarrow \sum\limits_{i\leftarrow0}^{Tasktab.nbTask} cptAvant[i] - cptApres[i]$
            \STATE Retire t de la periode
        \ENDFOR
        \STATE Place les Task dans l'ordre croissant de génance
    \ENDFOR
    \RETURN Tasktab
    \end{algorithmic}
\end{algorithm}

\newpage
\section{Conclusion}

Dans le cadre du module TER du S2 Master Informatique à l’UVSQ, nous avons eu l’occasion de réaliser un projet sous la direction de Mr Yann Strozecki et Mael Guiraud.
Nous avons choisi, parmi les sujets proposés, le sujet "Algorithme glouton de remplissage" car c'est une sujet qui demande une bonne compréhension de l'algorithmique ce qui nous a beaucoup intéressé.

De nos jour les échanges par les différents réseaux sont centralisés dans des datacenters ou cloud.
Pour gagner en efficacité il faut minimisé la latence lors de l'envoie d'un message vers un cloud.

L'objectif de ce projet est de concevoir et comparer des algorithmes gloutons qui permettent de placer au mieux des tâches periodiques avec des contraintes portant sur les paires de tâches.
Pour cela nous utilisons un modèle où les tâches sont envoyés periodiquement et le temps entre l'envoie et la reception est fixe.
Dans ce modèle il y a deux periodes de taille P, l'envoie du tâches est placé sur la première periode et la réception sur la seconde après un delai.
Il faut donc réussir a placer un maximum de tâches dans la periode.

\end{document}